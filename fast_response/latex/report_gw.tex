\documentclass[titlepage]{article}
\usepackage{tikz}
\usepackage{xcolor}
\usepackage{graphicx}
\usepackage{grffile}
\usepackage{longtable}
\usepackage[margin=1.0in]{geometry}

\setlength{\parindent}{0cm}
\newcommand{\sourcename}{IceCube 191001A}
\newcommand{\analysisid}{IceCube_191001A_2019_9_30}
\newcommand{\gfurate}{/mnt/lfs3/user/apizzuto/fast_response_skylab/from_gw/IceCube_191001A_2019_9_30/GFU_rate_plot.png}
\newcommand{\reportdate}{2019-10-02}
\newcommand{\skymap}{/mnt/lfs3/user/apizzuto/fast_response_skylab/from_gw/IceCube_191001A_2019_9_30/unblinded_skymap.png}
\newcommand{\skymapzoom}{/mnt/lfs3/user/apizzuto/fast_response_skylab/from_gw/IceCube_191001A_2019_9_30/unblinded_skymap_zoom.png}
\newcommand{\limitdNdE}{/mnt/lfs3/user/apizzuto/fast_response_skylab/from_gw/IceCube_191001A_2019_9_30/central_90_dNdE.png}
\newcommand{\muonfilter}{/mnt/lfs3/user/apizzuto/fast_response_skylab/from_gw/IceCube_191001A_2019_9_30/MuonFilter_13_plot.png}
\newcommand{\Lfilter}{/mnt/lfs3/user/apizzuto/fast_response_skylab/from_gw/IceCube_191001A_2019_9_30/OnlineL2Filter_17_plot.png}
\newcommand{\badnessplot}{/mnt/lfs3/user/apizzuto/fast_response_skylab/from_gw/IceCube_191001A_2019_9_30/badness_plot.png}
\newcommand{\tsd}{No background TS distribution}
\newcommand{\obsdate}{2019-09-30 -- 2019-10-02}
\newcommand{\multiplicity}{/mnt/lfs3/user/apizzuto/fast_response_skylab/from_gw/IceCube_191001A_2019_9_30/IN_ICE_SIMPLE_MULTIPLICITY_plot.png}
\newcommand{\survivialfunctionplot}{}
\newcommand{\backgroundpdfplot}{}
\newcommand{\sourcetable}{
\begin{longtable}{ll}
\hline
Source Name & IceCube 191001A\\
Trigger Time & 2019-10-01 20:09:18.170 (MJD=58757.839794)\\
Start Time & 2019-09-30 20:09:18.170 (Trigger-86400.0s)\\
Stop Time & 2019-10-02 20:09:18.170 (Trigger+86400.0s)\\
Time Window & 172800.0s\\
\hline
\end{longtable}}
\newcommand{\skylabtable}{
\begin{longtable}{ll}
\hline
Skylab Version & 2.7.2\\
IceTray Path & ['/data/user/apizzuto/software/icerec_V05_02/build/lib/icecube/icetray']\\
Created by & apizzuto\\
Dataset Used & /gfu online/version-001-p01/\\
Dataset details & Real-time Gamma-Ray Follow-Up (GFU) Sample with leap second bug fix and official\\
 & neutrino sources format.\\
\hline
\end{longtable}}
\newcommand{\runtimetable}{
\begin{longtable}{lllll}
\hline
Run & Start Time & Stop Time & Duration & Livetime\\ \hline
133115 & 2019-09-30 09:50:58 & 2019-09-30 17:50:59 & 8:00:00 & 0.0s\\
133116 & 2019-09-30 17:50:59 & 2019-10-01 01:51:06 & 8:00:07 & 20507.8s\\
133117 & 2019-10-01 01:51:06 & 2019-10-01 09:51:13 & 8:00:06 & 28807.0s\\
133118 & 2019-10-01 09:51:13 & 2019-10-01 17:51:23 & 8:00:10 & 28810.0s\\
133119 & 2019-10-01 17:52:26 & 2019-10-02 01:52:26 & 8:00:00 & 28800.0s\\
133120 & 2019-10-02 01:52:26 & 2019-10-02 09:52:36 & 8:00:09 & 28810.0s\\
133121 & 2019-10-02 09:52:36 & 2019-10-02 17:52:42 & 8:00:06 & 28806.0s\\
133122 & 2019-10-02 17:52:42 & 2019-10-02 21:30:24 & 3:37:42 & 8196.2s\\
\hline
\end{longtable}}
\newcommand{\runstatustable}{
\begin{longtable}{lllllll}
\hline
Run & Status & Light & Filter Mode & Run Mode & OK & GFU\\ \hline
133115 & SUCCESS & dark & PhysicsFiltering & PhysicsTrig & OK & 191\\
133116 & SUCCESS & dark & PhysicsFiltering & PhysicsTrig & OK & 227\\
133117 & SUCCESS & dark & PhysicsFiltering & PhysicsTrig & OK & 210\\
133118 & SUCCESS & dark & PhysicsFiltering & PhysicsTrig & OK & 180\\
133119 & SUCCESS & dark & PhysicsFiltering & PhysicsTrig & OK & 170\\
133120 & SUCCESS & dark & PhysicsFiltering & PhysicsTrig & OK & 188\\
133121 & SUCCESS & dark & PhysicsFiltering & PhysicsTrig & OK & 213\\
133122 & UNKNOWN & dark & PhysicsFiltering & PhysicsTrig & NotOK & 67\\
\hline
\end{longtable}}
\newcommand{\livetime}{172,737.0}
\newcommand{\ontimetable}{
\begin{longtable}{ll}
\hline
Access Method & database\\
Stream & \texttt{neutrino}\\
Query Time & 2019-10-02 20:30:24\\
Start Time & 2019-09-30 09:50:58\\
Stop Time & 2019-10-02 21:30:24\\
\hline
\end{longtable}}
\newcommand{\event}{[None]}\newcommand{\results}{
\begin{longtable}{ll}
\hline
$n_s$ & 0.000\\
$TS$ & 0.000\\
$P value$ & 1.0000\\
\hline
\end{longtable}}


\newcommand{\degree}{$^{\circ}$}
\begin{document}

\begin{titlepage}
  \centering
  \vspace{4cm}
  {\huge\bfseries IceCube GW Follow Up \\ Analysis Report\par}
  \vspace{1cm}
  {\LARGE For Internal Use Only\par}
  \vfill
  {\Large Source Name: \\ \itshape\sourcename\par}
  \vspace{0.5cm}
  {\Large Observation Date(s):\\ \obsdate \par}
  \vfill
  \vspace{1cm}
  {\Large Report Generated On:\\ \reportdate \par}
\end{titlepage}

\centering
\section{Inputs}
\subsection{Source Information}
\sourcetable
\subsection{On-Time Data}
\ontimetable
\subsection{Skylab Analysis Information}
\skylabtable 
% Information about skylab stable release, path to skylab, etc.

\pagebreak
\centering
\section{Detector Operations}

\subsection{Run Times}
\runtimetable

Total Livetime = \livetime\,s

\centering
\subsection{Run Status}
\runstatustable

\pagebreak
\subsection{Event Rates}
Plots for key trigger and filter rates for the data period
and the neighboring runs.  Red indicates selected time window.
Cyan indicates average GFU singlet rate. Badness $>10$s may indicate
a problem.

\vspace{1em}
{
 \centering
 \includegraphics[width=0.7\textwidth]{\multiplicity}\\
 \includegraphics[width=0.7\textwidth]{\muonfilter}\\
 \includegraphics[width=0.7\textwidth]{\Lfilter}\\
 \includegraphics[width=0.7\textwidth]{\gfurate}\\
 \includegraphics[width=0.7\textwidth]{\badnessplot}
}
\vspace{2cm}

\pagebreak

\section{Results}

{
  \centering
  {\Large All Sky On-time Events}

  \includegraphics[width=0.9\textwidth]{\skymap}
  \vspace{3cm}
  \includegraphics[width=0.9\textwidth]{\skymapzoom}
}

\begin{figure}
    \centering
    \decPDF
    \caption{Here we show IceCube's transient point source sensitivity in a 1000~s time window, overlaid with the declination PDF of the GW event. This is computed by binning the sky evenly in sin($\delta$) and computing the total probability in each bin. This is then converted to a PDF such that the blue curve above integrates to 1.}
\end{figure}

\newpage
\subsection{On-Time Events}
\event

\subsection{Likelihood Analysis}
\results

\newpage

\tsd

% \subsection{GFU Status}
% \GFUstatus
% \\
% \vspace{1cm}
% Here we summarize the status of the GFU stream during the time of the analysis.
% \\
% The first row shows the run-time of the analysis.
% The second row shows the number of events in the on-time window when the analysis initially ran.
% The third column checks the number of events in the on-time window after the analysis run time (first row) has passed. This number MUST BE THE SAME as the second row, otherwise something terrible has happened.
% The fourth row shows the number of GFU events that have arrived between the end of the on-time window and the end of the analysis run time. This shows that there are no extreme latency issues with the GFU stream.


\end{document}


